% Meta-Informationen -------------------------------------------------------
%    Informationen über das Dokument, wie z.B. Titel, Autor, Matrikelnr. etc
%    werden in der Datei _Meta.tex definiert und können danach global
%    verwendet werden.
% --------------------------------------------------------------------------
% Informationen ------------------------------------------------------------
%   Definition von globalen Parametern, die im gesamten Dokument verwendet
%   werden können (z.B auf dem Deckblatt etc.).
% --------------------------------------------------------------------------
\newcommand{\titel}{Use of Imitation Learning in an Obstacle Course Scenario}
\newcommand{\art}{Bachelor Thesis}
\newcommand{\ort}{Leipzig}
\newcommand{\hochschule}{Leipzig University}
\newcommand{\fachgebiet}{Database Department}
\newcommand{\fakultaet}{Faculty of Mathematics and Computer Science}
\newcommand{\institut}{Institute of Computer Science}
\newcommand{\autor}{Andrian Yarotskyi}
\newcommand{\matrikelnr}{3740921}
\newcommand{\erstbetreuer}{Dr. Thomas Burghardt}
\newcommand{\jahr}{2025}
\newcommand{\invnr}{1337}
\newcommand{\eingereicht}{27.05.2025}

% Eigene Befehle
\newcommand{\todo}[1]{\textbf{\textsc{\textcolor{red}{(TODO: #1)}}}}

% Autorennamen in small caps
\newcommand{\AutorZ}[1]{\textsc{#1}}
\newcommand{\Autor}[1]{\AutorZ{\citeauthor{#1}}}

% Befehle zur semantischen Auszeichnung von Text
\newcommand{\NeuerBegriff}[1]{\textbf{#1}}
\newcommand{\Fachbegriff}[1]{\textit{#1}}
\newcommand{\Prozess}[1]{\textit{#1}}
\newcommand{\Webservice}[1]{\textit{#1}}
\newcommand{\Eingabe}[1]{\texttt{#1}}
\newcommand{\Code}[1]{\texttt{#1}}
\newcommand{\Datei}[1]{\texttt{#1}}
\newcommand{\Datentyp}[1]{\textsf{#1}}
\newcommand{\XMLElement}[1]{\textsf{#1}}

% Abkürzungen
\newcommand{\vgl}{Vgl.\ }
\newcommand{\ua}{\mbox{u.\,a.\ }}
\newcommand{\zB}{\mbox{z.\,B.\ }}
\newcommand{\bs}{$\backslash$}

% Einfache Anführungszeichen in texttt
\newcommand{\sq}{\textquotesingle}


% Dokumentenkopf -----------------------------------------------------------
%   Diese Vorlage basiert auf "scrreprt" aus dem koma-script.
%    Die Option draft sollte beim fertigen Dokument ausgeschaltet werden.
% --------------------------------------------------------------------------
\documentclass[
  11pt,          % Schriftgröße
  DIV=10,
  english,        % für Umlaute, Silbentrennung etc.
  a4paper,        % Papierformat
  oneside,        % einseitiges Dokument
  titlepage,        % es wird eine Titelseite verwendet
  parskip=half,      % Abstand zwischen Absätzen (halbe Zeile)
  headings=normal, % Größe der Überschriften verkleinern
  numbers=withendperiod, % Fügt in den Überschriften nach den Zahlen einen Punkt ein
  listof=totoc,        % Verzeichnisse im Contentsverzeichnis aufführen
  bibliography=totoc,        % Literaturverzeichnis im Contentsverzeichnis aufführen
  index=totoc,        % Index im Contentsverzeichnis aufführen
  captions=tableheading,    % Beschriftung von Tabellen oberhalb ausgeben
  final          % Status des Dokuments (final/draft)
]{scrreprt}

\renewcommand*\chapterheadstartvskip{\vspace*{-1.0cm}}

% Bentigte Packages -------------------------------------------------------
%    Weitere Packages, die benötigt werden, sind in die Datei Packages.tex
%    "ausgelagert", um die Vorlage möglichst übersichtlich zu halten.
% --------------------------------------------------------------------------
\input{Packages}

% Erstellung eines Index und Abkürzungsverzeichnisses aktivieren -----------
\makeindex
% makeindex Bachelorarbeit.nlo -s nomencl.ist -o Bachelorarbeit.nls
\makenomenclature

% Kopf- und Fußzeilen, Seitenränder etc. -----------------------------------
\input{PageStyle.tex}

\begin{document}
% Eigene Definitionen für Silbentrennung
\include{Hyphenation}
% Das eigentliche Dokument -------------------------------------------------
%    Der eigentliche Content des Dokuments beginnt hier. Die einzelnen Seiten
%    und Kapitel werden in eigene Dateien ausgelagert und hier nur inkludiert.
% --------------------------------------------------------------------------
% auch subsubsection nummerieren
\setcounter{secnumdepth}{3}
\setcounter{tocdepth}{3}

% keine Kopf-/Fußzeilen bei Deckblatt und Abstract
\ofoot{}
% Deckblatt
\thispagestyle{plain}
\begin{titlepage}

  \begin{center}
    \includegraphics[height=7cm]{Bilder/Uni-L.png}\\[2.5ex]

    \hochschule\\
    \institut\\
    \fakultaet\\
    \fachgebiet\\[6ex]

    \textbf{\large\titel}\\[1.5ex]
    \art\\[6ex]

    \normalsize
    Submitted by:\\
    \autor\\[1.5ex]
    Matriculation number:\\
    \matrikelnr\\[1.5ex]
    Supervisor:\\
    \erstbetreuer\\
    \zweitbetreuer\\[1.0ex]
  \end{center}

  %\begin{tabbing}
  %\hspace{3.5cm}\= \kill
  %   vorgelegt von: \> \autor\\[1.2ex]
  %   Matrikelnummer: \> \matrikelnr\\[1.2ex]
  %    \> \\
  %   Betreuer: \> \erstbetreuer\\[1.2ex]
  %    \> \zweitbetreuer
  %\end{tabbing}

  \begin{center}
    \copyright\ \jahr\\[1.0ex]
  \end{center}

  \singlespacing
  \small
  \noindent This work including its parts is \textbf{protected by copyright}. Any use outside the narrow limits of copyright law without the consent of the author is prohibited and punishable. This applies in particular to duplications, translations, microfilming and storage and processing in electronic systems.

\end{titlepage}


\section*{Abstract}
\label{sec:Abstract}

This bachelor thesis investigates the limits of using Imitation Learning \autocite{10.1145/3054912} to train a Convolutional Neural Network that will be used in an agent that controls a self-driving system for completing an obstacle course. At the core of the system is a car-shaped JetBot robot with an artificial intelligence (AI) oriented board from NVIDIA. The main goal of the thesis is to contribute to the research of Imitation Learning in the domain of autonomous driving.

To assess the generalization abilities of the model the 2 levels of difficulties for the agent are introduced:
\begin{enumerate}
  \item \textbf{First level of difficulty:} In this setting the vehicle always starts from a certain position and the obstacles are also fixed at predetermined positions. This level of difficulty is used to test the basic skills of the model. As the environment remains constant, the robot can learn stable behavior through repeated training.
  \item \textbf{Second level of difficulty:} In this setting the positions of both the robot and the obstacles will vary in a random fashion. On this level the generalization abilities of the model will be tested. Since each drive is performed differently and not in a way, in which the model was trained, this difficulty level challenges the model's ability to spot and analyze more general features and not just to rely on memorization.
\end{enumerate}

The research is aimed to put clearness in the following questions: (1) Can the model be trained using Behavioral Cloning \autocite{8855753} to demonstrate human performance or even surpass it in the first difficulty? (2) Can the model be sufficiently generalized using behavioral cloning to demonstrate human performance or even surpass it in the second difficulty? (3) Can the Inverse Reinforcement Learning (IRL) \autocite{ng2000algorithms} \autocite{neu2012apprenticeshiplearningusinginverse} \autocites{lee2021approximateinversereinforcementlearning} approach eliminate the expected weaknesses of the Behavioral Cloning approach in mastering the course on both difficulty levels and contribute to successfully mastering the course? \\
The last question will be answered with a help of literature, since there is no possibility to implement IRL algorithms in current settings.

\ofoot{\pagemark}

% Seitennummerierung -------------------------------------------------------
%    Vor dem Hauptteil werden die Seiten in großen römischen Ziffern
%    nummeriert...
% --------------------------------------------------------------------------
\pagenumbering{Roman}

\tableofcontents      % Contentsverzeichnis

% Abkürzungsverzeichnis ----------------------------------------------------
%\input{Content/Glossar}
%\printnomenclature
%\label{sec:Glossar}

\listoffigures          % Abbildungsverzeichnis
\listoftables          % List of tables

%\renewcommand{\lstlistlistingname}{Verzeichnis der Listings}
%\lstlistoflistings

% ...danach in normalen arabischen Ziffern ---------------------------------
\clearpage
\pagenumbering{arabic}

% Content -------------------------------------------------------------------
%    Hier können jetzt die einzelnen Kapitel inkludiert werden. Sie müssen
%    in den entsprechenden .TEX-Dateien vorliegen. Die Dateinamen können
%     natürlich angepasst werden.
% --------------------------------------------------------------------------
\chapter{Introduction}
\label{cha:Introduction}

\section{Motivation}

This thesis is focused on research of Imitation Learning in the domain of autonomous driving and mostly targets the Behavioral Cloning technique and uses it as a main instrument. There have already been many successful attempts to train a Convolutional Neural Network using Imitation Learning to make controlling decisions in the scenario of an autonomous driving vehicle, both in virtualized \autocite{8855753} and real \autocite{pan2019agileautonomousdrivingusing} \autocite{bojarski2016endendlearningselfdriving} environments. Some of them are aimed for training the model to perform controls according to the trafficking rules in an urban setting. This development branch attracts researchers since there are many available Open Source datasets aimed for autonomous driving in an urban setting (e.g. A2D2 \autocite{geyer2020a2d2audiautonomousdriving}, Udacity Self Driving Car Dataset). Other researches set their goal to train the model to autonomously drive in a proprietary setting (e.g. aggressive driving using custom vehicles \autocite{drews2017aggressivedeepdrivingmodel}). Such researches are usually harder to make, since the task of collecting and labeling the data lies on the researchers.

\section{Goals of the thesis}

Since the thesis mostly focuses Behavioral Cloning, it tries to address the main weakness of this approach: maximize the model's generalization ability \autocite{ARGALL2009469}. Since it's learning from demonstration, the model learns a way to solve a problem and doesn't reinvent it from scratch, as in unsupervised machine learning approaches. This makes the learned behavior prone to undemonstrated states that didn't occur in training. And since the agent operates on it's own during tests, every it's decision potentially leads to an undemonstrated state. Based on this, it makes it very important to mainly focus on generalization of the model during training.

To assess the generalization abilities of the model the 2 levels of difficulties for the agent are introduced:
\begin{enumerate}
  \item \textbf{First level of difficulty:} In this setting the vehicle always starts from a certain position and the obstacles are also fixed at predetermined positions. This level of difficulty is used to test the basic skills of the model. As the environment remains constant, the robot can learn stable behavior through repeated training.
  \item \textbf{Second level of difficulty:} In this setting the positions of both the robot and the obstacles will vary in a random fashion. On this level the generalization abilities of the model will be tested. Since each drive is performed differently and not in a way in which the model was trained, this difficulty level challenges the model's ability to spot and analyze more general features and not just to rely on memorization.
\end{enumerate}

The research is aimed to put clearness in the following questions: (1) Can the model be trained using Behavioral Cloning \autocite{5152385} to demonstrate human performance or even surpass it in the first difficulty? (2) Can the model be sufficiently generalized using behavioral cloning to demonstrate human performance or even surpass it in the second difficulty? (3) Can the Inverse Reinforcement Learning (IRL) \autocite{ng2000algorithms} \autocite{neu2012apprenticeshiplearningusinginverse} \autocites{lee2021approximateinversereinforcementlearning} approach eliminate the expected weaknesses of the Behavioral Cloning approach in mastering the course on both difficulty levels and contribute to successfully mastering the course? \\
The last question will be answered with a help of literature, since there is no possibility to implement IRL algorithms in the given circumstances circumstances.

\section{Structure of the thesis}

\chapter{Background}
\label{cha:Background}

\section{Deep learning}

\begin{definition}
  An \textit{Artificial Neural Network (ANN)} \autocite{oshea2015introductionconvolutionalneuralnetworks} \autocite{sharma2017activation} is a computational model inspired by the structure and function of biological neural networks. It consists of interconnected artificial neurons (or nodes), which collectively process input data and learn patterns through training.

  In the case of Feedforward Neural Networks (FNNs), neurons are organized into layers, with each neuron in one layer connected to every neuron in the subsequent layer via directed connections. These connections are associated with trainable parameters called weights. Other neural architectures, such as Restricted Boltzmann Machines (RBMs) and Recurrent Neural Networks (RNNs), use different strategies for grouping and connecting neurons to capture specific data structures or temporal dependencies.

  Each neuron in the network also possesses a bias term—another learnable parameter—used to adjust the activation threshold. The output of a neuron is computed as a function of the weighted sum of its inputs and its bias, passed through a non-linear activation function. During training, both weights and biases are updated to minimize the network’s prediction error and improve its performance.
\end{definition}

\begin{definition}
  An \textit{Activation Function} \autocite{sharma2017activation} is a mathematical function used in artificial neural networks to determine the output of a neuron based on its input, associated weights, and bias. Given a neuron with inputs \(x\), weights \(w\), and bias \(b\), the neuron computes a linear combination \(z = w^\top x + b\). The activation function is then applied to this linear output, producing the final output of the neuron. Activation functions introduce non-linearity into the network, enabling it to learn complex patterns and approximate arbitrary functions. Common activation functions include the Sigmoid, Hyperbolic Tangent (Tanh), and Rectified Linear Unit (ReLU).
\end{definition}

An activation function is typically expected to exhibit two key mathematical properties:
\begin{enumerate}
  \item \textbf{Nonlinearity} \autocite{augustine2024surveyuniversalapproximationtheorems}: To satisfy the conditions of the Universal Approximation Theorem and enable the network to model complex, non-linear relationships, activation functions must introduce nonlinearity into the system. Using a nonlinear activation function in hidden layers is essential for learning intricate patterns in the data, but linear functions are also sometimes used, e.g. in the output layer of neural networks.
  \item \textbf{Differentiability} \autocite{sharma2017activation}: The activation function should ideally be differentiable across its entire domain to allow the computation of gradients during backpropagation. This is necessary for calculating loss gradients with respect to the network’s weights, enabling optimization techniques like Gradient Descent. However, it is not strictly necessary for the function to be continuously differentiable. For instance, the Rectified Linear Unit (ReLU) is not differentiable at zero, yet it is widely used—particularly in combination with convolutional layers—due to its empirical effectiveness and computational efficiency.
\end{enumerate}

\begin{definition}
  \textit{Deep supervised learning} \autocite{alzubaidi2021review} \autocite{cun2015deeplearning} \autocite{oshea2015introductionconvolutionalneuralnetworks} is a machine learning approach that relies on labeled data to train deep neural networks. Let \(D_t\), denote the training dataset, where each example \((x_t, y_t) \in D_t\) consists of an input datapoint \(x_t\) and its corresponding output label \(y_t\). A function \(f\), typically represented by an artificial neural network (ANN), is learned to map inputs to outputs. The discrepancy between the predicted output \(f(x_t)\) and the true output \(y_t\) is quantified using a loss function \(\gamma\), yielding a loss value \(\gamma(y_t, f(x_t))\). Optimization techniques such as Gradient Descent are then used to iteratively update the network’s parameters in order to minimize the loss, thereby improving the model’s accuracy in approximating the target function.
\end{definition}

\section{Convolutional Neural Networks}

Convolutional Neural Networks (CNNs) are one of the most commonly used tools for solving problems related to Computer Vision. They are widely used for classification, dimensionality reduction and other operations on high dimensional data. The ability to cost-efficiently store and derive features from image data without needing to store enormous amounts of weights is what makes them one of the most effective instruments for analyzing multidimensional data. Although CNNs are mostly used to process 2-dimensional data, they could also be used to word with 1D or 3D data in such tasks as sequence prediction or classification.

\begin{definition}
  A \textit{kernel} \autocite{alzubaidi2021review} is a grid of weights of a certain size and dimensionality (depends on the dimensionality of the CNN). It is used in convolutional layers when performing convolutional operation.
\end{definition}

\begin{definition}
  A \textit{convolutional operation} \autocite{alzubaidi2021review} is a process in which a kernel slides through the input image horizontally and vertically and dot product between kernel and input is calculated for each sliding point. The sliding and calculation of the dot product between matrices is performed until no further sliding is not possible. The sliding can be performed by visiting every pixel in each direction, but kernel can also slide skipping one or multiple pixels, depending on the stride size of the layer. The grid of scalar values that appear after performing convolutional operation is usually called a feature map.
\end{definition}

\begin{definition}
  A \textit{Convolutional Neural Network} \autocite{oshea2015introductionconvolutionalneuralnetworks} \autocite{jmse9040397} \autocite{LIU201711} is a discriminative deep learning model, which architecture is inspired by the organization of animal visual cortex. The main building blocks of every CNN are convolutional and sub-sampling/pooling layers. \\
  \textbf{Convolutional layers} consist of multiple kernels/filters, each of which has it's own set of weights and produces it's output by sliding through the input matrix and performing convolutional operation. Convolutional layers are used to extract features and figure positional relations between them. \\
  \textbf{Sub-sampling layers}
\end{definition}

The concept of CNNs was inspired by Time-Delay Neural Networks, where weights are shared through time dimension. In CNNs weights are compacted into it's kernels. This architectural property vastly reduces the number of parameters and complexity of the network compared to Fully Connected Networks. CNNs are valued for their ability to learn to identify features without human interaction and for their empirically proven performance in processing multidimensional data with grid-like topology, like images and videos.

\chapter{Related Work}
\label{cha:RelatedWork}

\section{Previous Work at ScaDS.AI}

This thesis builds directly upon the foundational work conducted by König \autocite{konig2022model}, Flach \autocite{flach2023methods}, Schaller \autocite{schaller2023train}, and Schneeberger \autocite{schneeberger2024end}, continuing the exploration of autonomous driving agents in the field of machine learning. König's thesis is the earliest work among them. His focus was on training a model to navigate through an obstacle course composed of block-like barriers in a simulated setting. The structure of his obstacle course closely resembles the one used in this work, with a key distinction: the present thesis standardizes the obstacles by making them uniformly red and arranging them in pairs, which requires the vehicle to consistently pass between them, adding a layer of spatial precision to the navigation challenge.

König \autocite{konig2022model} employed a reinforcement learning methodology, specifically leveraging an evolution-based training approach. In contrast to the approach taken in this thesis, König \autocite{konig2022model} utilized an OpenCV-based image processing pipeline to identify obstacles and arena boundaries within the input images. These visual elements were then encoded into tensors that directly represented their spatial properties. As a result, there was no need for a Convolutional Neural Network (CNN), since the network was not required to extract spatial features from raw image data. Instead, a standard Artificial Neural Network (ANN) architecture was sufficient for his setup. His results demonstrated that the model successfully learned to handle a variety of generated obstacle configurations, showing a high degree of generalization across different courses. He concluded that the trained agent was capable of completing most of the parkour-style tracks effectively.

The work of Flach \autocite{flach2023methods} was a direct continuation of König’s earlier research \autocite{konig2022model}, with a focus on addressing one of the most significant challenges in autonomous agent development: bridging the Simulation-to-Reality (Sim2Real) gap. While König concentrated on training the agent within a controlled simulated environment, Flach aimed to transfer the trained model to a real-world setting. This step was crucial for validating whether behaviors learned in simulation could generalize effectively to physical environments. Flach \autocite{flach2023methods} used the same JetBot vehicle and physical arena that are also employed in the current thesis.

However, the transition to the real environment introduced several technical difficulties. One of the primary challenges was the difference in control dynamics between the simulated and physical JetBot vehicles, which made direct transfer of control strategies problematic. Additionally, Flach relied on an OpenCV-based visual processing pipeline, similar to König, which proved to be unreliable under varying real-world lighting conditions. The pipeline often misinterpreted background objects as obstacles and was particularly sensitive to brightness fluctuations, leading to inconsistent performance. The inability of the pipeline to detect object consistently has lead to the refuse to continue the research with the real world experiments and need in usage of the synthesized data. All these problems and differences have lead to the failure of transitioning the model from simulation to reality.

Schaller \autocite{schaller2023train} extended the research done by König \autocite{konig2022model} in addressing the main inaccuracies in his approach and the main problems that occurred during the Flach's \autocite{flach2023methods} research. His work addressed four central research questions, focusing on the feasibility of modeling autonomous driving as an RL problem, effective processing of camera input, overall learning performance, and the robustness of the algorithms under external influences. To achieve this, he developed four RL algorithms using carefully designed state representations and reward functions. These models demonstrated strong performance, particularly on easy and medium tracks, with the PPO-MEM-SGT algorithm even managing to complete the most challenging courses. Instead of using a convolutional neural network, Schaller \autocite{schaller2023train} extracted key coordinates from camera images to construct the state input, showing that simpler preprocessing can yield efficient results.

Schneeberger \autocite{schneeberger2024end} focused on developing a robust autonomous driving agent capable of adapting to varying lighting conditions. His approach employed an end-to-end trained Convolutional Neural Network (CNN), enabling the agent to directly learn control policies from raw visual input without relying on handcrafted feature extraction or preprocessing pipelines. The driving task was structured as a sequence of navigation goals across three difficulty levels, each presenting increasingly complex spatial configurations. The agent was evaluated under three distinct lighting conditions (bright, standard, and dark) designed to test its adaptability and generalization in diverse visual environments.

The results demonstrated that the CNN-based agent performed reliably across all difficulty levels and lighting settings, indicating a high degree of robustness and learning efficiency. By successfully addressing the limitations posed by inconsistent illumination (a challenge that had hindered earlier works) Schneeberger’s \autocite{schneeberger2024end} research marked a significant step forward in the development of more resilient autonomous driving systems.

\chapter{Model Training}
\label{cha:Main}

\section{Environment Setup}

\subsection{Arena}

The arena used in this thesis can be represented as a $(2 \times 1)m$ surface, on which the obstacles with dimensions $(4 \times 4 \times 16) cm$ are installed. The setup resembles the one used in the thesis of König \autocite{konig2022model} with the distinction that in the current thesis all the obstacles are of one color. To achieve preciseness in measurements, stability and robustness of the material, it was decided to use the 3D-printing technology for the obstacle creating. Such approach also ensures that the process of transferring the setup into the virtual reality goes smoothly, in case there is a need for virtualization. For the first difficulty the obstacles were glued to the surface of the arena to ensure that they'll remain their exact position throughout preparing and evaluation processes. Due to a need in frequent changes in obstacle positions, in the case of second difficulty they were not glued. In every experiment run there are three obstacle pairs installed on the arena. Their position is deliberately chosen to increase the probability that they'd be in the sight of the camera the whole time before the robot passes them.

\subsection{Hardware}

To collect the training data and test the trained models we use the same JetBot by NVIDIA that was used in previous works from ScaDS.AI. It has $4$ GB of video-memory and $128$ CUDA-cores that are able to operate at the maximum of $921$ MHz clock speed. These computational abilities are enough for the system to be able to execute the final control prediction pipeline at the frequency of $4$ times per second. As for the moving part, it consists of 2 electro-motors connected to the computing unit. To control the mechanical part of the robot we used the JetBot API, which very conveniently has Python bindings. There are multiple modes in which one could operate the JetBot, but the most straightforward approach, which was used in this thesis, is to directly set the speeds of the motors to values in range $[-1,1]$.

\subsection{Software}
For the simplicity of making frequent changes to the model architecture, it was decided to use the Keras framework for Python programming language. This is a high-level framework that was built on top of existing Tensorflow library, which encapsulates all the GPU-computations and low-level details, simplifying the development to machine learning models. The main reasoning for choosing this framework were the simplicity of the syntax and a large quantity of ready-to-use layers, activation functions, learning optimizators etc. It also comes with very convenient toolset for neural network training, that includes embedded data augmentator for image datasets. Unfortunately, since dealing with the data sequences, which need to be handled very specifically in order to train the model using the right framerate, a custom data augmentation pipeline was developed.

There was a problem in matching the versions of tensorflow between different devices. The Jetson hardware supports only a particular version of cuDNN, which is required for Tensorflow. Therefore, there was a need in downgrading the version of the library for the whole project.

For dealing with large arrays and performing fast computations on matrices a NumPy Python library was utilized. It comes with a large number of mathematical functions that execute the compiled C++ code underneath, which gives such dynamic language as Python the fast computational advantages of native programming languages.

All the operations that were performed on images were done using the OpenCV Python library. This library comes with a large quantity of methods to transform, modify and store the image data. It was a very useful tool throughout the whole work on the thesis.

\section{Data Acquisition}

To be able to perform supervised training on a demonstration data (Behavioral Cloning), a labeled dataset needed. The dataset $D$ is represented as a vector where each entry $(x_i, y_i) \in D$ is a tuple, where $x_i$ is the data obtained from the sensors (camera input) and $y_i$ is the vector that represents the controls required for the vehicle to move at the current point in time. In this case, a gamepad was used to be able to collect the data and specifically the inclination of it's left stick are used to calculate the motor speeds of the JetBot. Specially for the acquisition part, a special software pipeline was developed, the main purpose of which was to enable the control of the vehicle, collection of data from the sensors and storing the collected data. The pipeline consists of two software components: \textit{server}, which is executed on the JetBot itself, and \textit{client}, which is executed on any other machine connected to the same network as JetBot. The developed pipeline works as follows:

\begin{enumerate}
  \item The server program on the JetBot is started. It consists of two separate threads.
  \item In one thread the OpenCV video capturing pipeline is started.
  \item In another thread a TCP server starts listening for a new connection on a particular port. Python's internal socket library was used to perform all the network operations.
  \item On another machine the client software is launched. This program ensures that the gamepad is connected to the machine. To obtain information about connected devices and listen for controlling inputs from the gamepad the Pygame library was used.
  \item When the connection is established, the server on JetBot awaits the input data from the port in a loop.
  \item Upon receiving the first controlling input from the gamepad the client software starts continuously sending the current position of the gamepad's stick to the server.
  \item Once the new data is received in the TCP-server-thread, the variable that is shared between both threads is updated.
  \item Once the video-thread captures a new frame, it is stored on the device. Then it checks for the updates in the controlling input's variable and then derives the motor inputs from this data. For a gamepad's stick position represented as $(x, y)$, the inputs for left and right motors are calculated as follows in the algorithm \ref{alg:motor_inputs}.
  \item Using JetBot API the values are passed to the motors and the cycle continues until the termination of the program.
  \item When the program is terminated, a process of transferring all of the sensor data from the JetBot to the client is started.
  \item At the moment the data transfer is finished, all the video frames are stored in a separate directory. The controlling sequence is stored in the same directory in a default format used by the NumPy Python library. The controls data is represented as a sequence of tuples $(x, y, t)$, where $x$ and $y$ encode gamepad's stick position and $t$ is the time in milliseconds from the start of the run.
\end{enumerate}

\begin{algorithm}
  \caption{Calculation of motor inputs based on the gamepad's stick position}
  \begin{algorithmic}[1]
    \Function{CalculateMotorInputs}{$x, y$}
    \State $\text{rotation\_quotient} \gets 0.5$
    \State $left\_power \gets -y + x \cdot \text{rotation\_quotient}$
    \State $right\_power \gets -y - x \cdot \text{rotation\_quotient}$
    \State $max\_power \gets \max\left( \left|left\_power\right|, \left|right\_power\right|, 1 \right)$
    \State $left\_power \gets left\_power / max\_power$
    \State $right\_power \gets right\_power / max\_power$
    \State \Return $left\_power, right\_power$
    \EndFunction
  \end{algorithmic}
  \label{alg:motor_inputs}
\end{algorithm}

The data was firstly collected using smaller frame rates (30FPS), but then it was decided to increase the frequency to 120FPS. The reasoning behind this was that, although the system is not able to execute the agent on such high frequencies, the frame sequence could be split while training and it would be possible to train the model using the frequency that suits the conditions.

Once all the data is collected, it should have ben filtered in the way that all the bad (unrepresentative) trajectories are deleted and the good trajectories are cropped to remove the data points that could obstruct the learning process of feature extraction. In particular, the image sequence that followed after the the JetBot crossing the last obstacle pair was completely deleted since it didn't contain any useful demonstration data.

The final dataset consists of $22017$ labeled images and is split into 2 parts: first difficulty and second difficulty. It must be said that the quality of the data for the second difficulty is not as good as for the first one and the quantity of the data is too small even for the first difficulty. The first difficulty part consists of $15008$ data-points and the rest if the second difficulty. Considering that the second difficulty requires a lot more generalizability from the model than the first one, the dataset for the second difficulty should've contained more data and it should've been more diverse and with better feature distribution variance. Due to time restrictions, the amount of work load that this thesis required and the time-consuming and difficult process of obtaining and filtering new data, it was decided to continue the training process using the existing dataset.

\section{Data Pre-processing}

Before starting the training with the data that was collected, the next step is to pre-process it to make training more effective and to reduce the amount of noise that the model could mishandle for a feature. The preprocessing steps for the images that were used in the final iteration of the model training process are as follows:

\begin{enumerate}
  \item \textbf{Gray-scaling} \\
    There is a hardware restriction on the time and space complexity of the agent that we're able to run on the JetBot. Thus, the size of the model and it's computational time have a big impact on the frequency with which the model will be able to predict the controls. For this purpose, all the images are gray-scaled before passing them into the network. The formula $Y = 0.299 \times R + 0.587 \times G + 0.114 \times B$, where $R$, $G$ and $B$ are the matrices of the red, green and blue components of the image respectively, was used to perform the gray-scaling operation.
  \item \textbf{Pixel-centering} \\
    For the CNN to achieve a good accuracy, it's not enough to train it on the raw data. As the experiments show \autocite{pal2016preprocessing}, the CNNs perform much better if the data they are fed is normalized in some way. In the final iteration of the agent all the images that are used as an input for the neural network are centralized right before passing them into the neural network using formula $Y_c = Y / 127.5 - 1$, where $Y$ is the gray-scaled image matrix.
  \item \textbf{Region of Interest} \\
    The region of interest  is cropped out of each image. This is done to reduce the dimensionality of the network, so that the computations are faster, and to increase the percentage of pixels that are important for feature extraction on the image.
\end{enumerate}

\subsection{Edge detection}
\label{sec:edge-detection}

Despite the pre-processing steps being straightforward and not convoluted in the final iteration of the agent, a huge work was done on experimenting with different approaches and figuring out, what works the best.

Farag et al. \autocite{8855753} came up with a sound future work proposition in their work: to utilize edge detection mechanism on an image before passing it into the CNN. On an intuitive level this approach would enable the CNN to be able to derive features from images more easily, since all the objects would be more distinguishable. Theoretically it would simplify the task of selecting obstacles on the way of the robot, so that the model would be able to lean to derive more complex and meaningful for the task features faster. Two edge detection algorithms were tested: one of them was the Canny edge detector \autocite{canny1986computational}, as proposed in \autocite{8855753}, another one was the Sobel filter \autocite{sobel2014history}. After training and experimenting with the models trained using both these approaches, there were no signs that edge detection would somehow help the model better generalize more complex features. The best results that could be obtained using edge detectors were for the JetBot to stably pass the second pair of obstacles and then fail to pass the third one. The first model trained on data that did not go through the process of edge detection has shown great performance improvement in comparison to it's predecessors. The main reason for such poor performance could be that the algorithms are very noise-prone. Although methods such as Gaussian blurring were used to smooth the images and decrease noise level, the algorithms still produced artifacts that the model was very likely to mismatch for a feature.

Despite edge detection being a very promising technique, the training was not able to achieve any good results using this approach. Another interesting development possibility was was to use Quantized Neural Network (QNN). In this method all the weights are represented as either integer values or even single-digit binary values. Since Canny edge detector algorithm produces binary matrix, it would've been possible to develop a quantized CNN that would derive features from such matrix. Such approach would drastically lower the size of the model and increase the computational speed, since integer-valued operations cost less computing cycles than operations with the floating point.

\section{Model Architecture}

One of the main tasks of this thesis was to develop an architecture of the CNN that would be able to learn to extract sufficient features from the image sequence data. Before starting the development of the architecture it had already been established that the model will have to be able to extract not only spacial features from the image, but also temporal features from the sequence of previous images. It has lead to the necessity to use a concept of memory stack \autocite{schneeberger2024end} \autocite{schaller2023train}.

Different configurations and approaches were tested to find the best model architecture that would be able to learn to extract features under the constraint of the poor quality of the data. In the scope of this work, tens of model architectures and their configurations were trained. Many of them didn't prove to be efficient in the task solving and were immediately put aside after testing. Here is the listing of the most performative models:

\begin{enumerate}
  \item \textbf{Previous theses' approach:} \\
    The approach of Schneeberger \autocite{schneeberger2024end} and Schaller \autocite{schaller2023train} consisted of using the convolutional channels to process the sequence --- one channel per image. For this purpose, the CNN model from the OpenAI's Stable Baselines 3 (the same model that was used in \autocite{schneeberger2024end}) was replicated using Keras (originally it was implemented in PyTorch) and slightly corrected to fit the dimensional requirements. The architecture is depicted in figure \ref{fig:sb3cnn-arch}.
    \begin{figure}[htbp]
      \centering
      \includegraphics[width=0.8\textwidth]{Images/SB3CNN_architecture.png}
      \caption{Architecture diagram of the CNN model from Stable Baselines 3.}
      \label{fig:sb3cnn-arch}
    \end{figure}
    The model didn't show well during the training --- because of the absence of improvement in the validation loss, the training was terminated automatically by the Early Stopping mechanism already on epoch $10$.
  \item \textbf{DAVE-2:} \\
    The architecture used by researchers from NVIDIA in \autocite{bojarski2016endendlearningselfdriving} was replicated exactly as it is in the paper. This is the only model that was used in this thesis that didn't use any kind of memory mechanism and simultaneously the only model that didn't require any kind of preprocessing to be done, except for image cropping. The model takes the raw image as an input (3-channeled input layer is used to process red, green and blue components of the image). The architecture is demonstrated in figure \ref{fig:dave-2-arch}.
    \begin{figure}[htbp]
      \centering
      \includegraphics[width=0.5\textwidth]{Images/DAVE-2_architecture.png}
      \caption{DAVE-2 architecture.}
      \label{fig:dave-2-arch}
    \end{figure}
  \item \textbf{BCNet with LSTM:} \\
    BCNet is the name of the architecture used in \autocite{8855753}. It resembles DAVE-2 in many ways, but has some significant distinctions (number of layers and layer sizes). The goal in \autocite{8855753} was to develop a model that would predict the steering angle of the vehicle. Certain corrections had to be made in order to fit this model into the setting of current thesis --- including dimensional changes in layers, output etc. Farag et al. \autocite{8855753} proposed multiple possible improvements for their system, including usage of Long Short-Term memory (LSTM) layers and edge detection mechanism for images before passing them into the CNN. As already stated in \autoref{sec:edge-detection}, edge detection hasn't proven to be a working solution that would somewhat improve the network's feature extraction abilities.

    The architecture of the model is illustrated in figure \ref{fig:BCNet-LSTM-arch}. In order to be able to combine the LSTM and normal architecture in one model, the the usage of Time Distributed layers from Keras was involved. These layers enable to perform transformation using convolutional layers on each of the images in the data sequence separately and then pass the transformed sequence to the LSTM in one batch.
    \begin{figure}[htbp]
      \centering
      \includegraphics[width=0.9\textwidth]{Images/BCNetLSTM_architecture.png}
      \caption{BCNet with LSTM layers. The Time Distributed layers represent the Convolutional, Flattening and Dropout layers of the network.}
      \label{fig:BCNet-LSTM-arch}
    \end{figure}
\end{enumerate}

The detailed explanation of performance testing and comparison between models is described in \autoref{cha:Evaluation}.

% TODO: model architecture, model diagram, image preprocessing diagram

% Literaturverzeichnis -----------------------------------------------------
%    Das Literaturverzeichnis wird aus der Datenbank erstellt.
%    Die genaue Verwendung von biblatex wird hier jedoch nicht erklärt.
%    Links:   https://ctan.org/pkg/biblatex?lang=de
%            https://de.overleaf.com/learn/latex/Articles/Getting_started_with_BibLaTeX
% --------------------------------------------------------------------------

\printbibliography

% \setcounter{page}{122}
% \pagenumbering{gobble}
%\pagenumbering{gobble}
\addchap{Declaration of Authorship}
I do solemnly declare that I have written the presented research thesis:

\begin{center}
  \textit{\glqq\titel\grqq}\\[1em]
\end{center}

by myself without undue help from a second person others and without using such tools
other than that specified. Where I have used thoughts from external sources, directly or
indirectly, published or unpublished, this is always clearly attributed. I am aware that
infringement can also subsequently lead to the cancellation of the degree.
\par
\ort, the \eingereicht

\rule[-0.2cm]{5cm}{0.5pt}

\textsc{\autor}
  % Selbständigkeitserklärung

% Anhang -------------------------------------------------------------------
%    Die Contente des Anhangs werden analog zu den Kapiteln inkludiert.
%    Dies geschieht in der Datei Anhang.tex
% --------------------------------------------------------------------------
\appendix
\clearpage
\renewcommand*{\thesection}{\Alph{section}}
\pagenumbering{Roman}
%\include{Content/Anhang}

% Index --------------------------------------------------------------------
%    Zum Erstellen eines Index, die folgende Zeile auskommentieren.
% --------------------------------------------------------------------------
%\printindex    % Index hier einfügen
%\ofoot{}
%\include{Content/Thesen}  % Thesen

\end{document}
