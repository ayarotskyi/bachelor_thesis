\section*{Abstract}
\label{sec:Abstract}

This bachelor thesis investigates the limits of using Imitation Learning to train a Convolutional Neural Network that will be used in an agent that controls a self-driving system for completing an obstacle course. At the core of the system is a car-shaped JetBot robot with an artificial intelligence (AI) oriented board from NVIDIA. The main goal of the thesis is to contribute to the research of Imitation Learning in the domain of autonomous driving.

To assess the generalization abilities of the model the 2 levels of difficulties for the agent are introduced:
\begin{enumerate}
  \item \textbf{First level of difficulty:} In this setting the vehicle always starts from a certain position and the obstacles are also fixed at predetermined positions. This level of difficulty is used to test the basic skills of the model. As the environment remains constant, the robot can learn stable behavior through repeated training.
  \item \textbf{Second level of difficulty:} In this setting the positions of both the robot and the obstacles will vary in a random fashion. On this level the generalization abilities of the model will be tested. Since each run will be commited in a random fashion, % TODO: finish
\end{enumerate}

The research is aimed to put clearness in the following questions: (1) % TODO: ask Thomas if I can redefine questions