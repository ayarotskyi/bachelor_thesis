\chapter{Evaluation}
\label{cha:evaluation}

In this chapter, the performance of the trained behavioral cloning model (BCNet) is quantitatively and qualitatively compared to that of the expert control system. The evaluation is based on trajectory-level behavior analysis under controlled conditions, with specific attention to trajectory similarity, movement characteristics, and clustering structure.

\section{Experimental Setup}

To conduct a fair and objective evaluation, an overhead camera was installed above the arena to record the JetBot's movement during navigation tasks. The camera captured the complete trajectory of the robot in the $x$-$y$ plane for both the expert controller and the BCNet-driven controller.

Each trajectory represents a single run of the robot from start to completion of a predefined navigation task. More than 100 successful runs were collected for both the expert and the BCNet model. The recorded video data was manually post-processed to extract discrete trajectory coordinates over time, resulting in a dataset suitable for further analysis and metric computation.

\section{Trajectory Similarity Metrics}

To assess the similarity between the expert and neural network behaviors, several standard clustering and trajectory analysis metrics were applied. The two sets of trajectories generated by the expert and by the BCNet were treated as distinct clusters.

\subsection{Davies–Bouldin Index}

The \textbf{Davies–Bouldin Index (DBI)} is a widely-used clustering metric that evaluates intra-cluster similarity and inter-cluster separation \autocite{4766909}. A lower DBI indicates that the clusters are compact and well-separated. The DBI is calculated as:

\[
  \text{DBI} = \frac{1}{k} \sum_{i=1}^{k} \max_{j \ne i} \left( \frac{\sigma_i + \sigma_j}{d(c_i, c_j)} \right)
\]

Where $\sigma_i$ and $\sigma_j$ are the average distances of points in clusters $i$ and $j$ to their respective centroids, and $d(c_i, c_j)$ is the distance between cluster centroids. In this case, trajectory similarity was calculated using Dynamic Time Warping (DTW), which is more appropriate than Euclidean distance for comparing temporal sequences of varying length.

\textbf{Result:} The DBI for the expert and BCNet trajectories was \textbf{1.779}. This value suggests moderate cluster overlap, implying that the model-generated trajectories share similar global structure with those of the expert, though some separation remains.

\subsection{Silhouette Score}

The \textbf{Silhouette Score} \autocite{kaufman2009finding} measures how similar an object is to its own cluster (cohesion) compared to other clusters (separation). It ranges from $-1$ to $1$, with higher values indicating better-defined clusters. It is defined as:

\[
  s(i) = \frac{b(i) - a(i)}{\max(a(i), b(i))}
\]

Where $a(i)$ is the mean intra-cluster distance and $b(i)$ is the mean nearest-cluster distance for sample $i$.

To better understand directional movement accuracy, silhouette scores were calculated independently for the $x$ and $y$ axes of the trajectories:

\begin{itemize}
  \item \textbf{X-Axis Silhouette Score:} $0.49$ — This relatively high score indicates that the BCNet hasn't succeeded in mimicking the forward-driving behavior of the expert.
  \item \textbf{Y-Axis Silhouette Score:} $0.32$ — This lower value reflects more precision in side-to-side movement, which is associated with steering and turning maneuvers.
\end{itemize}

\begin{figure}[htbp]
  \centering
  \includegraphics[width=0.6\textwidth]{Images/Evaluation/trajectory_overlay.png}
  \caption{Overlaying plot of the trajectories.}
  \label{fig:trajectory_overlay}
\end{figure}

\subsection{Mean Velocity Comparison}

Comparison between velocity profiles of the expert and the trained model is represented in \autoref{fig:velocity_profiles}. The average linear velocity of the JetBot throughout the whole trajectory was also analyzed:

\begin{itemize}
  \item \textbf{Mean velocity of BCNet:} $0.14$ m/s
  \item \textbf{Mean velocity of Expert:} $0.17$ m/s
\end{itemize}

This result shows that while the BCNet-driven JetBot was capable of navigating the environment, it tended to move more cautiously (i.e., slower) than the expert. This behavior is typical in behavioral cloning approaches, especially in cases of errors during training or poorly constructed dataset \autocite{bühler2020drivingghostsbehavioralcloning}.

\begin{figure}[H]
  \centering
  \includegraphics[width=0.8\textwidth]{Images/Evaluation/velocity_profiles.png}
  \caption{Relation between mean velocity and time.}
  \label{fig:velocity_profiles}
\end{figure}

\section{Summary of Evaluation Metrics}

\begin{table}[H]
  \centering
  \caption{Trajectory Evaluation Metrics}
  \label{tab:evaluation_metrics}
  \begin{tabular}{|l|c|}
    \hline
    \textbf{Metric} & \textbf{Value} \\
    \hline
    Davies–Bouldin Index & 1.779 \\
    Silhouette Score (X-axis) & 0.49 \\
    Silhouette Score (Y-axis) & 0.32 \\
    Mean Velocity (BCNet) & 0.14 m/s \\
    Mean Velocity (Expert) & 0.17 m/s \\
    \hline
  \end{tabular}
\end{table}