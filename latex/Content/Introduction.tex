\chapter{Introduction}
\label{cha:Introduction}

\section{Motivation}

This thesis is focused on research of Imitation Learning in the domain of autonomous driving and mostly targets the Behavioral Cloning technique and uses it as a main instrument. There have already been many successful attempts to train a Convolutional Neural Network using Imitation Learning to make controlling decisions in the scenario of an autonomous driving vehicle, both in virtualized \autocite{8855753} and real \autocite{pan2019agileautonomousdrivingusing} \autocite{bojarski2016endendlearningselfdriving} environments. Some of them are aimed at training the model to perform controls according to the trafficking rules in an urban setting. This development branch attracts researchers since there are many available Open Source datasets based on autonomous driving in an urban setting (e.g. A2D2 \autocite{geyer2020a2d2audiautonomousdriving}, Udacity Self Driving Car Dataset). Other researches set their goal to train the model to autonomously drive in a proprietary setting (e.g. aggressive driving using custom vehicles \autocite{drews2017aggressivedeepdrivingmodel}). Such researches are usually harder to make, since the task of collecting and labeling the data lies on the researchers.

\section{Goals of the thesis}

Since the thesis mostly focuses Behavioral Cloning, it tries to address the main weakness of this approach: maximize the model's generalization ability \autocite{ARGALL2009469}. Since it's learning from demonstration, the model learns a way to solve a problem and doesn't reinvent it from scratch, as in unsupervised machine learning approaches. This makes the learned behavior prone to undemonstrated states that didn't occur in training. And since the control inference system operates on it's own during tests, every its decision potentially leads to an undemonstrated state. Based on this, it makes it very important to mainly focus on generalization of the model during training.

To assess the generalization abilities of the model the two levels of difficulties for the control inference system are introduced:
\begin{enumerate}
  \item \textbf{First level of difficulty:} In this setting the vehicle always starts from a certain position and the obstacles are also fixed at predetermined positions. This level of difficulty is used to test the basic skills of the model. As the environment remains constant, the robot can learn stable behavior through repeated training.
  \item \textbf{Second level of difficulty:} In this setting the positions of both the robot and the obstacles will vary in a random fashion. On this level the generalization abilities of the model will be tested. Since each drive is performed differently and not in a way in which the model was trained, this difficulty level challenges the model's ability to spot and analyze more general features and not just to rely on memorization.
\end{enumerate}

The research is aimed to put clearness in the following questions: (1) Can the model be trained using Behavioral Cloning \autocite{5152385} to demonstrate human performance or even surpass it in the first difficulty? (2) Can the model be sufficiently generalized using behavioral cloning to demonstrate human performance or even surpass it in the second difficulty? (3) Can the Inverse Reinforcement Learning (IRL) \autocite{ng2000algorithms} \autocite{neu2012apprenticeshiplearningusinginverse} \autocites{lee2021approximateinversereinforcementlearning} approach eliminate the expected weaknesses of the Behavioral Cloning approach in mastering the course on both difficulty levels? \\
The last question will be answered with a help of literature, since there is no possibility to implement IRL algorithms in the given circumstances.

\section{Structure of the thesis}

This thesis progresses from theoretical foundations to practical implementation and validation. Beginning with core concepts in deep learning and imitation learning, it develops a complete end-to-end training pipeline featuring environment setup, data processing, model training and the real-time control inference system that translates predictions into driving actions. Then the work delves into the experimental validation across difficulty levels, followed by conclusions that connect empirical results to broader autonomous driving challenges.