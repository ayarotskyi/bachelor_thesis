\chapter{Related Work}
\label{cha:RelatedWork}

\section{Previous Work at ScaDS.AI}

This thesis builds directly upon the foundational work conducted by König \autocite{konig2022model}, Flach \autocite{flach2023methods}, Schaller \autocite{schaller2023train}, and Schneeberger \autocite{schneeberger2024end}, continuing the exploration of autonomous driving agents in the field of machine learning. König's thesis is the earliest work among them. His focus was on training a model to navigate through an obstacle course composed of block-like barriers in a simulated setting. The structure of his obstacle course closely resembles the one used in this work, with a key distinction: the present thesis standardizes the obstacles by making them uniformly red and arranging them in pairs, which requires the vehicle to consistently pass between them, adding a layer of spatial precision to the navigation challenge.

König \autocite{konig2022model} employed a reinforcement learning methodology, specifically leveraging an evolution-based training approach. In contrast to the approach taken in this thesis, König \autocite{konig2022model} utilized an OpenCV-based image processing pipeline to identify obstacles and arena boundaries within the input images. These visual elements were then encoded into tensors that directly represented their spatial properties. As a result, there was no need for a Convolutional Neural Network (CNN), since the network was not required to extract spatial features from raw image data. Instead, a standard Artificial Neural Network (ANN) architecture was sufficient for his setup. His results demonstrated that the model successfully learned to handle a variety of generated obstacle configurations, showing a high degree of generalization across different courses. He concluded that the trained agent was capable of completing most of the parkour-style tracks effectively.

The work of Flach \autocite{flach2023methods} was a direct continuation of König’s earlier research \autocite{konig2022model}, with a focus on addressing one of the most significant challenges in autonomous agent development: bridging the Simulation-to-Reality (Sim2Real) gap. While König concentrated on training the agent within a controlled simulated environment, Flach aimed to transfer the trained model to a real-world setting. This step was crucial for validating whether behaviors learned in simulation could generalize effectively to physical environments. Flach \autocite{flach2023methods} used the same JetBot vehicle and physical arena that are also employed in the current thesis.

However, the transition to the real environment introduced several technical difficulties. One of the primary challenges was the difference in control dynamics between the simulated and physical JetBot vehicles, which made direct transfer of control strategies problematic. Additionally, Flach relied on an OpenCV-based visual processing pipeline, similar to König, which proved to be unreliable under varying real-world lighting conditions. The pipeline often misinterpreted background objects as obstacles and was particularly sensitive to brightness fluctuations, leading to inconsistent performance. The inability of the pipeline to detect object consistently has lead to the refuse to continue the research with the real world experiments and need in usage of the synthesized data. All these problems and differences have lead to the failure of transitioning the model from simulation to reality.

Schaller \autocite{schaller2023train} extended the research done by König \autocite{konig2022model} in addressing the main inaccuracies in his approach and the main problems that occurred during the Flach's \autocite{flach2023methods} research. His work addressed four central research questions, focusing on the feasibility of modeling autonomous driving as an RL problem, effective processing of camera input, overall learning performance, and the robustness of the algorithms under external influences. To achieve this, he developed four RL algorithms using carefully designed state representations and reward functions. These models demonstrated strong performance, particularly on easy and medium tracks, with the PPO-MEM-SGT algorithm even managing to complete the most challenging courses. Instead of using a convolutional neural network, Schaller \autocite{schaller2023train} extracted key coordinates from camera images to construct the state input, showing that simpler preprocessing can yield efficient results.

Schneeberger \autocite{schneeberger2024end} focused on developing a robust autonomous driving agent capable of adapting to varying lighting conditions. His approach employed an end-to-end trained Convolutional Neural Network (CNN), enabling the agent to directly learn control policies from raw visual input without relying on handcrafted feature extraction or preprocessing pipelines. The driving task was structured as a sequence of navigation goals across three difficulty levels, each presenting increasingly complex spatial configurations. The agent was evaluated under three distinct lighting conditions (bright, standard, and dark) designed to test its adaptability and generalization in diverse visual environments.

The results demonstrated that the CNN-based agent performed reliably across all difficulty levels and lighting settings, indicating a high degree of robustness and learning efficiency. By successfully addressing the limitations posed by inconsistent illumination (a challenge that had hindered earlier works) Schneeberger’s \autocite{schneeberger2024end} research marked a significant step forward in the development of more resilient autonomous driving systems.